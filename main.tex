\documentclass[12pt]{article}

\usepackage{sbc-template}

\usepackage{graphicx,url}

%\usepackage[brazil]{babel}
\usepackage[utf8]{inputenc}

\usepackage{amsmath}
\usepackage{physics}
\usepackage{algpseudocode}
\usepackage{algorithm}

\sloppy

\title{On Explaining Neural Networks outputs using Linear Optimization}

\author{Tiago Vargas Pereira de Oliveira\inst{1}, Thiago Alves Rocha\inst{1}}

\address{Instituto Federal de Educação, Ciência e Tecnologia do Ceará (IFCE) \email{tiago.vargas06@aluno.ifce.edu.br, thiago.alves@ifce.edu.br}}

\begin{document}

\maketitle

\begin{abstract}
	This meta-paper describes the style to be used in articles and short papers for SBC conferences. In both cases, abstracts should not have more than 10 lines and must be in the first page of the paper.
\end{abstract}

\section{Introduction}

\section{Preliminaries}

\section{Improving Minimal Explanations}

% TODO: Talk about how to improve an explanation given an minimal explanation

% TODO: Talk about sufficient and necessary explanations

Suppose we have a neural network classifier with inputs $x_1$, $x_2$, $x_3$, $x_4$, $x_5$, and possible outputs $C_1$, $C_2$, $C_3$.

Suppose, after passing $\vb{x} = \begin{bmatrix} x_1 \\ x_2 \\ x_3 \\ x_4 \\ x_5 \end{bmatrix} = \begin{bmatrix} 0.89 \\ -1.07 \\ 0.42 \\ -0.61 \\ -0.67 \end{bmatrix}$ to the classifier, we get the class $C_2$, with minimal explanation as follows:

\[
M = \{x_1 = 0.89, x_3 = 0.42, x_4 = -0.61\}
\]

In this case, $x_2$ and $x_5$ are not part of the minimal explanation, meaning they can have any value and the output will still be $C_2$, as long as $x_1 = 0.89$, $x_3 = 0.42$, $x_4 = -0.61$ still holds.

We are now interested in discovering which range of values for the variables $x_1$, $x_3$ and $x_4$ that still guarantees the output to be $C_2$, i.e. we are insterested in improving the minimal explanation to consist of constraints of type $c_{i-} \le x_i \le c_{i+}$.

The approach is to see if the problem is solvable when walking at steps of $\epsilon$ at the vicinity of the point.
To achieve this, we initially treat each constraint of type $x = c$ as $c \le x \le c$, and check if the problem is solvable, after each increment of size $\epsilon$ on each side, one side at time.
We expect the improved explanation to be a set of constraints that describe the range of each variable in the minimal explanation, such as $c - k_{-} \cdot \epsilon \le x \le c + k_{+} \cdot \epsilon$, where $k_-$ is the number of steps taken to the left, and $k_+$ is the number of steps taken to the right.

\begin{algorithm}
	\caption{Improve the minimal explanation by steps}
	\begin{algorithmic}
		\Procedure{GetImprovedExplanation} {$M, F, C, \epsilon$}
			\ForAll {$c \in M$}
				\While {$M \cup F \models C$}
					\State {$M' \gets M$}
					\State {$c.rhs \gets c.rhs + \epsilon$}
				\EndWhile
				\State {$M \gets M'$}

				\While {$M \cup F \models C$}
					\State {$M' \gets M$}
					\State {$c.lhs \gets c.lhs - \epsilon$}
				\EndWhile
				\State {$M \gets M'$}
			\EndFor
		\EndProcedure
	\end{algorithmic}
\end{algorithm}

% TODO: Talk about gaps in the interval


\section{Experiments}

% TODO: Talk about that dataset $\{A \ge 0.5 \implies 1\}$ and mention how `anchor` fails to find the boundary (0.5)

% TODO: Talk about the execution time

% TODO: Talk about the correctness


\section{Conclusion}

\section{References}

% TODO: Add actual citations

% Citations from template:
Bibliographic references must be unambiguous and uniform.  We recommend giving
the author names references in brackets, e.g. \cite{knuth:84},
\cite{boulic:91}, and \cite{smith:99}.


\bibliographystyle{sbc}
\bibliography{sbc-template}

\end{document}
