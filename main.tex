\documentclass[12pt]{article}

\usepackage{sbc-template}

\usepackage{graphicx,url}

%\usepackage[brazil]{babel}
\usepackage[utf8]{inputenc}

\usepackage{amsmath}
\usepackage{physics}
\usepackage{algpseudocode}
\usepackage{algorithm}
\usepackage{tikz}
\usepackage{xcolor}

\usetikzlibrary{positioning, matrix}

\sloppy

\title{On improving minimal explanations for neural networks using mixed integer linear programming}

\author{Tiago Vargas Pereira de Oliveira\inst{1}, Thiago Alves Rocha\inst{1}}

\address{Instituto Federal de Educação, Ciência e Tecnologia do Ceará (IFCE) \email{tiago.vargas06@aluno.ifce.edu.br, thiago.alves@ifce.edu.br}}

\begin{document}

\maketitle

\begin{abstract}
	% TODO: Add abstract in no more than 10 lines
\end{abstract}

\section{Introduction}

Neural networks have become widely used in a variety of applications due to their ability to learn complex patterns from large amounts of data and make accurate predictions.
However, neural networks are usually black-box models, giving little insight into how they arrive at their predictions.
This lack of transparency is problematic if the model's prediction contradicts the opinion of experts, because it is difficult to know if the model has identified a new pattern or simply has made a mistake.
In cases where wrong decisions can lead to severe consequences, as in deciding to buy or sell stocks, it becomes particularly difficult to know who to trust.
By developing explainers for neural networks, experts can reason through the decision-making process of the model and decide to either reject or accept its prediction more confidently.

% TODO: Talk about logic-based explainers

Most approaches for obtaining neural network explanations are based on heuristic methods, which means they do not guarantee correctness.
Well-known examples are LIME \cite{ribeiro2016should}, Anchors \cite{Ribeiro_Singh_Guestrin_2018} and SHAP \cite{lundberg2017unified}.
These heuristic approaches do not consider the entire feature space, and as a result, the explanations provided by these methods may be incorrect. In other words, there may be inputs for which the classification and the explainer disagree.
% Feature space | input space | [possible] inputs

There are neural network explainers that work with the concept of \emph{minimal explanation}, which is a set of which features and their corresponding values that contributed to the model's prediction.
We aim to improve minimal explanations by identifying the \emph{range} of values that a feature can take, rather than just the value it assumes in the input.


\section{Preliminaries}

% TODO: Talk about mixed integer linear programming

% TODO: Talk about logical consequences (mention proof by contradiction on outputting the class)

% TODO: Talk about how to model a minimal explanation from a dataset
A dataset is a collection of features and their respective values.
We represent a dataset by a set of constraints of type $x_i = c_i$, to indicate that feature $x_i$ has value $c_i$.
% QUESTION: Is it really a set of constraints?

% TODO: Talk about neural network notations

Neural networks are composed of layers of interconnected neurons, where each neuron holds a value and each connection has a weight.
The neurons in one layer are connected to every neuron in the next layer.
We can think of a neuron as a function that takes the values of all the neurons in the previous layer and generates a value.

Neurons can either be active or inactive. In this paper

% The idea is that given a pattern of values of the neurons in the first layer, the network outputs a pattern of values in the neurons in the last layer.

The bias tells us how high the weighted sum needs to be before the neuron starts getting meaningfully active.

\definecolor{apricot}{HTML}{FFCDB2}
\definecolor{melon}{HTML}{FFB4A2}
\definecolor{salmon pink}{HTML}{E5989B}
\definecolor{old rose}{HTML}{B5838D}
\definecolor{dim gray}{HTML}{6D6875}

\tikzset{
	neuron/.style={draw, circle, row sep=0.5cm},
	layer/.style={matrix of nodes, nodes=neuron},
	arrow/.style={->, dim gray}
}

\begin{tikzpicture}
	\matrix[layer, label=$L_0$, nodes={fill=salmon pink}] (input) {
		$x_1$ \\
		$x_2$ \\
		$x_3$ \\
		$x_4$ \\
		$x_5$ \\
	};

	\matrix[layer, label=$L_1$, right=of input, nodes={fill=apricot}] (hidden 1) {
		$x_1^1$ \\
		$x_2^1$ \\
		$x_3^1$ \\
		$x_4^1$ \\
	};

	\matrix[layer, label=$L_2$, right=of hidden 1, nodes={fill=apricot}] (hidden 2) {
		$x_1^2$ \\
		$x_2^2$ \\
		$x_3^2$ \\
		$x_4^2$ \\
	};

	\matrix[layer, label=$L_3$, right=of hidden 2, nodes={fill=old rose}] (output) {
		$o_1$ \\
		$o_2$ \\
		$o_3$ \\
	};

	% Connecting the neurons of neighbor layers
	\foreach \i in {1, ..., 5} {
		\foreach \j in {1, ..., 4} {
			\draw[arrow] (input-\i-1) -- (hidden 1-\j-1);
		}
	}

	\foreach \i in {1, ..., 4} {
		\foreach \j in {1, ..., 4} {
			\draw[arrow] (hidden 1-\i-1) -- (hidden 2-\j-1);
		}
	}

	\foreach \i in {1, ..., 4} {
		\foreach \j in {1, 2, 3} {
			\draw[arrow] (hidden 2-\i-1) -- (output-\j-1);
		}
	}
\end{tikzpicture}

\begin{align*}
	x_1^{(1)} &= \sigma(w_{1,1}^{(1)} x_1^{(0)} + w_{1,2}^{(1)} x_2^{(0)} + w_{1,3}^{(1)} x_3^{(0)} + w_{1,4}^{(1)} x_4^{(0)} + b_1^{(0)}) \\
	          &= \sigma(\sum_{i=1}^{4}{w_{1,i}^{(1)} x_i^{(0)}} + b_1^{(0)}) \\
	\vb{x}^{(1)} &= \sigma(\vb{W}^{(1)}\vb{x}^{(0)} + \vb{b}^{(0)})
\end{align*}


\section{Improving Minimal Explanations}

% TODO: Talk about how to improve an explanation given a minimal explanation

% TODO: Talk about sufficient and necessary explanations

Suppose we have a neural network classifier with inputs $x_1$, $x_2$, $x_3$, $x_4$, $x_5$, and possible outputs $C_1$, $C_2$, $C_3$.

Suppose, after passing $\vb{x} = \begin{bmatrix} x_1 \\ x_2 \\ x_3 \\ x_4 \\ x_5 \end{bmatrix} = \begin{bmatrix} 0.89 \\ -1.07 \\ 0.42 \\ -0.61 \\ -0.67 \end{bmatrix}$ to the classifier, we get the class $C_2$ as output, with minimal explanation $M$ as follows:

\[
M = \{x_1 = 0.89, x_3 = 0.42, x_4 = -0.61\}
\]

In this case, $x_2$ and $x_5$ are not part of the minimal explanation, meaning they can have any value and the output will still be $C_2$, as long as $x_1 = 0.89$, $x_3 = 0.42$, $x_4 = -0.61$ still holds.

We are now interested in discovering which range of values for the features $x_1$, $x_3$ and $x_4$ still guarantees the output to be $C_2$, i.e. we are interested in improving the minimal explanation to consist of constraints of type $c_{i-} \le x_i \le c_{i+}$.

The approach is to see if the output is still the same after walking at steps of $\epsilon$ at the vicinity of the point.
To achieve this, we iterate over $M$, treating the current constraint of type $x = c$ as $c \le x \le c$.
We then expand the range of the constraint to include values in the vicinity to the right of $c$ by updating it to be $c \le x \le c + \epsilon$, and check if this updated $M$ still guarantees the class $C_2$.
We keep incrementing $\epsilon$ to the right-hand side of the constraint until we cannot guarantee the output class to remain the same, in which case the previous value is the upper end of the interval for feature $x$.
After that, we do an analogous procedure for the left side of the constraint, but decrementing at steps of $\epsilon$, until we find the lower end for $x$.
In the case no steps can be taken at all, we keep the constraint as $x = c$.
In the case we reach a value outside the domain of $x$, we limit the constraint to include values up to that boundary.
% TODO: [on the last sentence] Explain better that we cap the range **after** checking the output out

We expect the improved explanation to be a set of constraints that describe the range of each feature in the minimal explanation, such as $c - k_{-} \cdot \epsilon \le x \le c + k_{+} \cdot \epsilon$, where $k_{-}$ is the number of steps taken to the left, and $k_{+}$ is the number of steps taken to the right.

We provide an algorithm for improving a minimal explanation below. The line $M' \gets M$ means that $M'$ is a \emph{deep copy} of $M$.

In the context of the previous example, suppose we aim to stretch the interval for $x_1$, first finding the upper end $0.92$, after choosing a step size of $\epsilon = 0.1$:

\[
	M = \{0.89 \le x_1 \le 0.92,
	      x_3 =  0.42,
	      x_4 = -0.61\}
\]

Then we stretch the lower end of the interval, finding $0.14$:

\[
	M = \{0.14 \le x_1 \le 0.92,
	      x_3 =  0.42,
	      x_4 = -0.61\}
\]

Having found the range for the input $x_1$, we move on to $x_3$, where we can't step to the right:

\[
	M = \{0.14 \le x_1 \le 0.92,
	      0.42 \le x_3 \le 0.42,
	      x_4 = -0.61\}
\]

Then we step to the left and find its lower end to be -0.51:

\[
	M = \{ 0.14 \le x_1 \le 0.92,
	      -0.51 \le x_3 \le 0.42,
	      x_4 = -0.61\}
\]

And finally, we move to $x_4$, where we can't step at all:

\[
	M = \{ 0.14 \le x_1 \le 0.92,
	      -0.51 \le x_3 \le 0.42,
	      x_4 = -0.61\}
\]

% TODO: Explain the notation $M \cup F \models O$

% TODO: Describe parameters

% M: Minimal explanation
% F: Set of logical formulas that describes the network
% O: Formula that describes which neuron is the greatest

\begin{algorithm}
	\caption{Improve the minimal explanation by steps}
	\begin{algorithmic}
		\Procedure{GetImprovedExplanation} {$M, F, O, \epsilon$}
			\ForAll {$c \in M$}
				\While {$M \cup F \models O$}
					\State {$M' \gets M$}
					\State {$c \texttt{.right\_hand\_side} \gets c \texttt{.right\_hand\_side} + \epsilon$}
				\EndWhile
				\State {$M \gets M'$}
				\\
				\While {$M \cup F \models O$}
					\State {$M' \gets M$}
					\State {$c \texttt{.left\_hand\_side} \gets c \texttt{.left\_hand\_side} - \epsilon$}
				\EndWhile
				\State {$M \gets M'$}
			\EndFor
			\\
			\State {\textbf{return} $M$}
		\EndProcedure
	\end{algorithmic}
\end{algorithm}

% TODO: Talk about gaps in the interval

% TODO: Put the example of the sliced square (2 inputs; 2 classes; one line dividing the square, defining the class boundaries; select a point and walk until you hit the line and notice how first walking vertically or horizontally changes the results)

\tikzset{
	axis/.style={black, thick, ->, >=stealth},
	rectangle/.style={draw=black!75, fill=gray!50, thick},
	line/.style={black, thick},
	movement range/.style={black!75, dashed, <->, >=latex, thick},
	figure/.style={scale=5, domain=\xmin:\xmax+\margin},
}

\def\margin{0.1}
\def\ymin{-\margin}
\def\ymax{\rectHeight + \margin}
\def\xmin{-\margin}
\def\xmax{\rectLength + \margin}

\newcommand{\drawAxes}{
	\draw[axis] (0, \ymin) -- (0, \ymax + \margin/2) node[above] {$x_2$};
	\draw[axis] (\xmin, 0) -- (\xmax + \margin/2, 0) node[right] {$x_1$};
}

\def\rectHeight{0.75}
\def\rectLength{1}

\newcommand{\drawRectangle}{
	\filldraw[rectangle] (0, 0) rectangle (\rectLength, \rectHeight);
}

\newcommand{\drawBoundary}{
	\draw[line] plot ({\x}, {\rectHeight - \x});
}

\newcommand{\drawPoint}[2]{
	\filldraw[black!80] #1 circle (0.5pt) node[above] {#2};
}

% Figure 1
\begin{tikzpicture}[figure]
	\drawAxes
	\clip (\xmin, \ymin) rectangle (\xmax, \ymax);
	\drawRectangle
	\drawBoundary

	% Colors
	\coordinate (A) at (\xmin, \rectHeight + \margin);  % Top left, line r
	\coordinate (B) at (\rectHeight + \margin, \ymin);  % Bottom right, line r
	\fill[opacity=0.25, color=orange] (\xmin, \ymin) -- (A) -- (B) -- cycle;
	\fill[opacity=0.25, color=blue] (A) -- (\rectLength + \margin, \rectHeight + \margin) -- (\rectLength + \margin, \ymin) --  (B) -- cycle;

	\def\Xx{0.8}
	\def\Xy{0.5}
	\def\X{(\Xx, \Xy)}
	\drawPoint{\X}{$\vb{x}$}

	% Dashed lines
	\draw[movement range] (\rectHeight - \Xy, \Xy) -- (\rectLength, \Xy);
\end{tikzpicture}

\begin{tikzpicture}[figure]
	\drawAxes
	\clip (\xmin, \ymin) rectangle (\xmax, \ymax);
	\drawRectangle
	\drawBoundary

	% Colors
	\coordinate (A) at (\xmin, \rectHeight + \margin);  % Top left, line r
	\coordinate (B) at (\rectHeight + \margin, \ymin);  % Bottom right, line r
	\fill[opacity=0.25, color=orange] (\xmin, \ymin) -- (A) -- (B) -- cycle;
	\fill[opacity=0.25, color=blue] (A) -- (\rectLength + \margin, \rectHeight + \margin) -- (\rectLength + \margin, \ymin) --  (B) -- cycle;

	% The Point
	\def\Xy{0.5}
	\def\Xx{\rectHeight - \Xy}
	\def\X{(\Xx, \Xy)}
	% \drawPoint{\X}{$\vb{x}$}
	\filldraw[black!80] (\Xx, \Xy) circle (0.5pt) node[below left] {$\vb{x}$};

	% % Dashed lines
	% Horizontal dashed line
	\draw[movement range] (\rectHeight - \Xy, \Xy) -- (\rectLength, \Xy);
	% Vertical dashed line
	\draw[movement range] (\Xx, \Xy) -- (\Xx, \rectHeight);

	% X movement area
	\fill[opacity=0.25, blue] (\rectHeight - \Xy, \Xy) rectangle (\rectLength, \rectHeight);
\end{tikzpicture}


% Figure 2
\begin{tikzpicture}[figure]
	\drawAxes
	\clip (\xmin, \ymin) rectangle (\xmax, \ymax);
	\drawRectangle
	\drawBoundary

	% Colors
	\coordinate (A) at (\xmin, \rectHeight + \margin);  % Top left, line r
	\coordinate (B) at (\rectHeight + \margin, \ymin);  % Bottom right, line r
	\fill[opacity=0.25, color=orange] (\xmin, \ymin) -- (A) -- (B) -- cycle;
	\fill[opacity=0.25, color=blue] (A) -- (\rectLength + \margin, \rectHeight + \margin) -- (\rectLength + \margin, \ymin) --  (B) -- cycle;

	% The Point
	\def\Xx{0.8}
	\def\Xy{0.5}
	\def\X{(\Xx, \Xy)}
	\drawPoint{\X}{$\vb{x}$}

	\def\step{0.1}

	% % Dashed lines
	% Upper dashed line
	\draw[movement range, ->] (\Xx, \Xy) -- ++(\step, 0) -- ++(0, \step) -- ++(\step, 0) -- (\rectLength, \rectHeight);

	% X movement area
	\fill[opacity=0.25, blue] (\Xx, \Xy) rectangle (\rectLength, \rectHeight);
\end{tikzpicture}

\begin{tikzpicture}[figure]
	\drawAxes
	\clip (\xmin, \ymin) rectangle (\xmax, \ymax);
	\drawRectangle
	\drawBoundary

	% Colors
	\coordinate (A) at (\xmin, \rectHeight + \margin);  % Top left, line r
	\coordinate (B) at (\rectHeight + \margin, \ymin);  % Bottom right, line r
	\fill[opacity=0.25, color=orange] (\xmin, \ymin) -- (A) -- (B) -- cycle;
	\fill[opacity=0.25, color=blue] (A) -- (\rectLength + \margin, \rectHeight + \margin) -- (\rectLength + \margin, \ymin) --  (B) -- cycle;

	% The Point
	\def\Xx{0.8}
	\def\Xy{0.5}
	\def\X{(\Xx, \Xy)}
	\drawPoint{\X}{$\vb{x}$}

	\def\step{0.1}

	% % Dashed lines
	% Upper dashed line
	\draw[movement range, ->] (\Xx, \Xy) -- ++(\step, 0) -- ++(0, \step) -- ++(\step, 0) -- (\rectLength, \rectHeight);
	% Lower dashed line
	\draw[movement range, ->] (\Xx, \Xy) -- ++(-\step, 0) -- ++(0, -\step) -- ++(-\step, 0) -- ++(0, -\step) -- ++(-\step, 0) -- (\Xx - 3*\step, \rectHeight - \Xx + 3*\step);

	% X movement area
	\fill[opacity=0.25, blue] (\Xx - 3*\step, \rectHeight - \Xx + 3*\step) rectangle (\rectLength, \rectHeight);
\end{tikzpicture}

% TODO: Put an alternative algorithm to do 1 step per variable, as opposed to discovering the range for 1 variable and then moving to the next one


\section{Experiments}

% TODO: Fazer tabela com...
% - Tempo Anchors V Alg1 V Alg2
% - "Volume" | "Cobertura" da explicação
% - Médias dos tamanhos da ranges
% - Número de features não generalizadas

% TODO: Talk about that dataset $\{A \ge 0.5 \implies 1\}$ and mention how `anchor` fails to find the boundary (0.5)

% TODO: Talk about the execution time

% TODO: Talk about the correctness


\section{Conclusion}


\bibliographystyle{sbc}
\bibliography{sbc-template}

\end{document}
